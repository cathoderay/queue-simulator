\chapter{Introdução}
\section{Funcionamento geral}
O simulador possui uma lista de eventos que é processada continuamente, até alcançar um número máximo de clientes que desejamos atender.
Essa lista de eventos ...

\section{Estruturas internas utilizadas}
Para viabilizar a implementação da ideia geral apresentada acima, dividimos o simulador em alguns módulos, a saber:

\section{Linguagem de Programação}
Para a codificação do simulador foi utilizada a linguagem de programação Python, versão 2.5.5.

\section{Geração de variáveis aleatórias}
A linguagem Python utiliza o gerador de números aleatórios ``Mersenne Twister", um dos métodos mais extensivamente testados existentes. 

O método garante que a sequência de números gerados pela chamada random() só se repetirá em um período de $2^{19937}-1$. Como o período é bem extenso, não precisamos nos preocupar com redefinir seeds que gerassem sequências sobrepostas.

A semente inicial utilizada pelo gerador, por default, é o timestamp corrente no momento do import do módulo random.

\section{Métodos utilizados}
Foi utilizado o método replicativo para a simulação.

\section{Implementação do conceito de cores}
• Indicar como implementou o conceito de cores ou equivalência;

\section{Escolha dos parâmetros}
• Explicação da escolha dos parâmetros utilizados nas rodadas da simulação e tabela com
os valores utilizados para cada cenário e para cada utilização (número de fregueses
coletados por rodada, número de rodadas, tamanho da fase transiente, etc.). Estes dados
podem também serem apresentados a cada resultado da simulação do item 4.

\section{Máquina utilizada}
Para a simulação utilizamos uma máquina com as seguintes configurações:
\begin{itemize}
  \item Processador: Intel Core Duo 2 GHz 
  \item Memória: 2GiB DDR 2 667
  \item Sistema Operacional: MAC OS X 10.5.8
\end{itemize}

• Indique a máquina utilizada e a duração dos experimentos que levaram ao fator mínimo
(explicado a frente)

\section{Informações pertinentes}
• Outras informações pertinentes
