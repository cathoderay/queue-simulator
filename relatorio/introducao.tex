\chapter{Introdução}
\section{Funcionamento geral}

O simulador possui uma lista de eventos que é processada continuamente, até alcançar um número máximo de clientes que desejamos atender por rodada.

São executadas tantas rodadas quanto forem necessárias até todos os intervalos de confiança dos valores que estão sendo estimados forem válidos, ou seja, $<=10\%$ da média do estimador.

Inicialmente, calculamos o tempo de chegada do primeiro cliente que representa um evento de chegada no sistema. A passsagem de um cliente pelo sistema possibilita a criação dos seguintes eventos:
\begin{itemize}
\item <tempo, tipo: chegada no sistema>
\item <tempo, tipo: entrada no servidor pela primeira vez>
\item <tempo, tipo: saída do servidor>
\item <tempo, tipo: entrada no servidor pela segunda vez>
\end{itemize}

Quando um evento de chegada ocorre, outro evento de chegada é criado com o tempo definido com o tempo de chegada baseado em uma distribuição exponencial, que representa o tempo de chegada do próximo cliente. Deste modo, os clientes vão chegando no sistema e a lista de eventos é processada.

Quando um evento é processado, ele é removido da lista de eventos e os novos eventos gerados a partir deste são criados e adicionados na lista, ordenada pelos tempos em que cada evento ocorre.

Todos os parâmetros, descritos na seção \ref{sec:parametros} são passados para o simulador em sua inicialização.

\section{Estruturas internas utilizadas}
Para viabilizar a implementação da ideia geral apresentada acima, dividimos o simulador em alguns módulos, abaixo estão explicitados os mais importantes:\\

\textbf{Módulos utilitários:}

\begin{itemize}
  \item Estimator: Módulo que possui métodos para retornar os estimadores de média, variância e calcula intervalos de confiança.
  \item Dist: Módulo com o método que retorna os tempos aleatórios de chegada de uma distribuição exponencial.
\end{itemize}

\textbf{Classes:}

\begin{itemize}
  \item Client: classe que representa um cliente que entra no sistema. Possui seus tempos de entrada e saída da fila, tempo no servidor e cor.
  \item EventHeap: classe que representa a lista de eventos que é processada durante uma rodada de simulação.
  \item Simulator: classe que implementa a lógica principal do simulador, processa as rodadas tratando os eventos e as chegadas dos clientes. E calcula as estimativas das variáveis aleatórias.
  \item Analytic: classe que serve para calcular os resultados de forma analítica.
\end{itemize}

\section{Linguagem de Programação}
Para a codificação do simulador foi utilizada a linguagem de programação Python, versão 2.5.5.

\section{Geração de variáveis aleatórias}
A linguagem Python utiliza o gerador de números aleatórios "Mersenne Twister", um dos métodos mais extensivamente testados existentes. 

O método garante que a sequência de números gerados pela chamada random() só se repetirá em um período de $2^{19937}-1$. Como o período é bem extenso, não precisamos nos preocupar com redefinir seeds que gerassem sequências sobrepostas.

A semente inicial utilizada pelo gerador, por default, é o timestamp corrente no momento do import do módulo random.

\section{Métodos utilizados}
Foi utilizado o método replicativo para a simulação.

\section{Implementação do conceito de cores}
O conceito de cores foi implementando adicionando o atributo ``color" no objeto Client, que possui 2 valores: TRANSIENT ou EQUILIBRIUM.  O número de clientes que representam a fase transiente são associados à cor TRANSIENT e os outros clientes são associados à cor EQUILIBRIUM.

Ao final da rodada de simulação os clientes que possuem a cor TRANSIENT são descartados do cálculo dos estimadores.

\section{Escolha dos parâmetros}
\label{sec:parametros}
Ao iniciar o simulador, são executados em sequência todas as simulações necessárias para obtermos todos os dados requeridos para ambas as políticas de atendimento com os parâmetros:
\begin{itemize}
  \item $\rho=0.2$ - \# de clientes na frase transiente = 30000
  \item $\rho=0.4$ - \# de clientes na frase transiente = 40000
  \item $\rho=0.6$ - \# de clientes na frase transiente = 80000
  \item $\rho=0.8$ - \# de clientes na frase transiente = 400000
  \item $\rho=0.9$ - \# de clientes na frase transiente = 500000
\end{itemize}

A escolha do número de clientes da fase transiente para cada utilização foi estimada de acordo com o que é exposto no capítulo \ref{chap:estimativa}.

O número de clientes processados a cada rodada é um parâmetro de entrada para o simulador. Para o cálculo dos resultados foram utilizados apenas os dados de 100000 clientes, sem contar os presentes na fase transiente.

\pagebreak
\section{Máquina utilizada}
As configurações da máquina utilizada para executar a simulação e os tempos de cada experimento são mostrados abaixo:\\

\textbf{Configurações:}

\begin{itemize}
  \item Processador: Intel Core Duo 2 GHz 
  \item Memória: 2GB DDR 2 667Mhz
  \item Sistema Operacional: MAC OS X 10.5.8 (Leopard)
\end{itemize}

\textbf{Duração dos experimentos:}

\begin{itemize}
  \item $\rho=0.2$ - F.C.F.S : 24.88s.
  \item $\rho=0.2$ - L.C.F.S : 19.15s.
  \item $\rho=0.4$ - F.C.F.S : 68.93s.
  \item $\rho=0.4$ - L.C.F.S : 27.58s.
  \item $\rho=0.6$ - F.C.F.S : 120.42s.
  \item $\rho=0.6$ - L.C.F.S : 27.97s.
  \item $\rho=0.8$ - F.C.F.S : 627.45s.
  \item $\rho=0.8$ - L.C.F.S : 1037.33s.
  \item $\rho=0.9$ - F.C.F.S : 3403.95s.
  \item $\rho=0.9$ - L.C.F.S : 4732.28s.
\end{itemize}
