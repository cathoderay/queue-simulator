\chapter{Estimativa da fase transiente}
\label{chap:estimativa}

Nesta seção você descreverá como a fase transiente foi estimada para os diversos valores de
r (obviamente existe um caso mais crítico). A fase transiente deve sempre implicar num
certo número de eventos de partida que são desprezados, esperando que o sistema entre em
equilíbrio. Este número de partidas em cada cenário e para cada valor de utilização deve ser
documentado, qualquer que seja o método escolhido para determinar o fim da fase
transiente.
A determinação da fase transiente é obrigatória, pois é um exercício para determinar a
entrada em equilíbrio do sistema. Você terá que justificar suas escolhas. Este é um processo
empírico.
Apresente resultados quantitativos que justificam sua escolha. Se você usou o método
batch, além da estimação da fase transiente, mostre como as estatísticas entre as rodadas
foram coletadas.
Procure demonstrar a influência da escolha da fase transiente na qualidade das medidas.
É preciso indicar com clareza se a estimativa utilizada é a mesma para os diferentes
cenários e diferentes valores da utilização. A determinação da fase transiente deve ser
independente da semente inicial. Comprove isso.
