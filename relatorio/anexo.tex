\lstset{
  language=Python,
  breaklines=true,
  numbers=left,
  numberstyle=\footnotesize,
  stepnumber=1,
  numbersep=5pt
}

\chapter{Listagem documentada do programa}

Este capítulo conterá o código fonte do simulador, dividido por tipo de módulo e ordenados por importância.\\

\section{Classes}

\subsection{Simulator}
Classe que implementa a lógica principal do simulador, processa as rodadas tratando os eventos e as chegadas dos clientes. E calcula as estimativas das variáveis aleatórias.\\

\begin{lstlisting}
import sys, random, math
from collections import deque
from util.constants import *
from util.progress_bar import ProgressBar
from util import estimator as est
from util import dist
from client import *
from event_heap import *


class Simulator:

    # Inicializacao do simulador
    def __init__(self, entry_rate, warm_up, service_policy, clients, server_rate=1.0, test=False):
        # Numero total de clientes = fase transiente + clientes a serem avaliados 
        self.total_clients = warm_up + clients
        self.samples = 1
        self.server_rate = server_rate
        self.entry_rate = entry_rate
        self.warm_up = warm_up
        # Definido aqui o metodo a ser utilizado para retirar os clientes da fila e coloca-los no servidor,
        # dependendo da politica de atendimento usada.
        if service_policy == FCFS:
            Simulator.__dict__['pop_queue1'] = Simulator.pop_queue1_fcfs
            Simulator.__dict__['pop_queue2'] = Simulator.pop_queue2_fcfs
            self.service_policy = 'First Come First Served (FCFS)'
        elif service_policy == LCFS:
            Simulator.__dict__['pop_queue1'] = Simulator.pop_queue1_lcfs
            Simulator.__dict__['pop_queue2'] = Simulator.pop_queue2_lcfs
            self.service_policy = 'Last Come First Served (LCFS)'
        self.init_sample()
        # Define o dicionario que ira guardar a soma e a soma dos quadrados das medias e variancias estimadas a cada rodada.
        self.sums = { 'm_s_W1': 0, 'm_s_s_W1': 0, 'v_s_W1': 0, 'v_s_s_W1': 0,            
                      'm_s_N1': 0, 'm_s_s_N1': 0, 'm_s_Nq1': 0, 'm_s_s_Nq1': 0,            
                      'm_s_T1': 0, 'm_s_s_T1': 0, 'm_s_W2': 0, 'm_s_s_W2': 0,            
                      'v_s_W2': 0, 'v_s_s_W2': 0, 'm_s_N2': 0, 'm_s_s_N2': 0,
                      'm_s_Nq2': 0, 'm_s_s_Nq2': 0, 'm_s_T2': 0, 'm_s_s_T2': 0 }
        # Dicionario que ira guardar os resultados de cada estimador calculados pelo simulador.
        self.results = {}
        # Define a lista com os clientes que serao testados, caso o simulador esteja em modo de teste.
        self.test = test
        self.test_list = []
        if self.test:
            self.init_test()
    
    # Inicializa as estruturas de dados para cada rodada
    def init_sample(self):
        # Filas do sistema.
        self.queue1 = deque([])
        self.queue2 = deque([])
        # Cliente que esta no servidor ( Quando esta variavel for nula significa que o servidor esta ocioso )
        self.server_current_client = None
        # Lista dos clientes que entraram no sistema durante a rodada.
        self.clients = []
        # Dicionario com a soma das variaveis que indicam o numero de pessoas nas filas (N) e em espera (Nq)
        self.N_samples = { 'Nq_1': 0, 'N_1': 0, 'Nq_2': 0, 'N_2': 0 }
        self.warm_up_sample = self.warm_up
        # Tempo do simulador.
        self.t = 0.0
        # Tempo do evento anterior ao que esta sendo processado.
        self.previous_event_time = 0.0
        # Lista de eventos.
        self.events = EventHeap()
        # Inicializa o simulador com o evento de chegada do primeiro cliente ao sistema.
        self.events.push((dist.exp_time(self.entry_rate), INCOMING))
    
    # Inicia o simulador
    def start(self):
        # Inicializa a barra usada para medir o progresso do simulador
        # Ela e contabilizada de acordo com o valor do menor intervalo de confianca encontrado a cada rodada,
        # Chegando a 100% quando o intervalo chega a 10% da media do estimador
        # Mostrada quando o simulador nao esta definido na forma de teste
        if not(self.test):
            prog = ProgressBar(0, 0.9, 77, mode='fixed', char='#')
            print "Processando as rodadas:"
            print prog, '\r',
            sys.stdout.flush()
        
        # Loop principal do simulador.
        # Termina quando todos os intervalos de confianca forem menores que 10% da media do estimador.
        while not(self.valid_confidence_interval()):
            # Loop de cada rodada, processa um evento a cada iteracao.
            while len(self.clients) <= self.total_clients:
                self.process_event()
            self.discard_clients()
            # Processa os dados gerados por uma rodada.
            self.process_sample()
            if self.samples > 1:
                self.calc_results()
                prog.update_amount(max(prog.amount, self.pb_amount()))
                print prog, '\r',
            self.samples += 1
            sys.stdout.flush()
            # Linha para forcar o teste a executar apenas 1 rodada.
            if self.test:
                break
        print
    
    # Metodo que processa um evento
    def process_event(self):
        # Remove um evento da lista para ser processado e atualiza o tempo do simulador
        self.t, event_type = self.events.pop()

        # Evento do tipo: Chegada ao sistema.
        if event_type == INCOMING:
            self.update_n()
            # Define a cor do cliente, verificando se ele chegou durante a fase transiente ou nao.
            if self.warm_up_sample > 0:
                new_client = Client(len(self.clients), TRANSIENT)
                self.warm_up_sample -= 1
            else:
                new_client = Client(len(self.clients), EQUILIBRIUM)
            # Adiciona o cliente na fila 1 e define o seu tempo de chegada nessa fila.
            new_client.set_queue(1)
            new_client.set_arrival(self.t)
            self.queue1.append(new_client)
            self.clients.append(new_client)
            # Teste de correcao
            if self.test and (new_client.id in self.test_list):
                print "Cliente", new_client.id, "gerou o evento Chegada ao sistema."
                print "Cliente", new_client.id, "entrou na fila 1."
            # Assim que uma chegada e processada, adiciona outro evento de chegada, dando o tempo que ela ira ocorrer.
            self.events.push((self.t + dist.exp_time(self.entry_rate), INCOMING))
            # Se o servidor estiver ocioso, adiciona o evento Entrada ao servidor pela fila 1 para esse cliente na lista.
            if not self.server_current_client:
                self.events.push((self.t, SERVER_1_IN))

        # Evento do tipo: Entrada ao servidor pela fila 1.
        elif event_type == SERVER_1_IN:
            # Define o tempo que o cliente vai ficar no servidor.
            server_time = dist.exp_time(self.server_rate)
            # Adiciona o cliente no servidor e define o seu tempo de saida da fila 1.
            self.server_current_client = self.pop_queue1()
            self.server_current_client.set_leave(self.t)
            self.server_current_client.set_server(server_time)
            # Teste de correcao
            if self.test and (self.server_current_client.id in self.test_list):
                print "Cliente", self.server_current_client.id, "gerou o evento Entrada ao servidor pela fila 1."
                print "Cliente", self.server_current_client.id, "entrou no servidor."
            # Adiciona o evento Saida do servidor na lista.
            self.events.push((self.t + server_time, SERVER_OUT))        

        # Evento do tipo: Entrada ao servidor pela fila 2.
        elif event_type == SERVER_2_IN:
            # Define o tempo que o cliente vai ficar no servidor.        
            server_time = dist.exp_time(self.server_rate)
            # Adiciona o cliente no servidor e define o seu tempo de saida da fila 2.
            self.server_current_client = self.pop_queue2()
            self.server_current_client.set_leave(self.t)
            self.server_current_client.set_server(server_time)
            # Teste de correcao
            if self.test and (self.server_current_client.id in self.test_list):
                print "Cliente", self.server_current_client.id, "gerou o evento Entrada ao servidor pela fila 2."
                print "Cliente", self.server_current_client.id, "entrou no servidor."
            # Adiciona o evento Saida do servidor na lista.
            self.events.push((self.t + server_time, SERVER_OUT))                

        # Evento do tipo: Saida do servidor.
        elif event_type == SERVER_OUT:
            self.update_n()
            # Se a fila 1 possuir clientes, adiciona o evento Entrada ao servidor pela fila 1 na lista.
            if self.queue1:
                self.events.push((self.t, SERVER_1_IN))
            # Se a fila 2 possuir clientes e a fila 1 vazia, ou se o sistema estiver vazio e o cliente que
            # esta no servidor entrou nele pela fila 1, adiciona o evento Entrada ao servidor pela fila 2 na lista.
            elif self.queue2 or self.server_current_client.queue == 1:
                self.events.push((self.t, SERVER_2_IN))
            
            # Teste de correcao
            if self.test and (self.server_current_client.id in self.test_list):
                print "Cliente", self.server_current_client.id, "gerou o evento Saida do servidor."
            
            # Se o cliente que esta no servidor entrou nele pela fila 1, adiciona ele na fila 2.
            if self.server_current_client.queue == 1:
                self.queue_2_in()
            # Senao, define que ele foi servido e saiu do sistema.
            else:
                self.server_current_client.set_served(1)
            self.server_current_client = None

    # Metodo que trata a entrada de um cliente na fila 2. Encapsulado para melhor legibilidade.
    def queue_2_in(self):
        # Pega o cliente do servidor e o adiciona na fila 2, definindo seu tempo de chegada na mesma.
        client = self.server_current_client
        self.queue2.append(client)
        client.set_queue(2)
        client.set_arrival(self.t)
        # Teste de correcao
        if self.test and (client.id in self.test_list):
            print "Cliente", client.id, "entrou na fila 2."

    # Atualiza o numero de pessoas nas filas a cada chegada, e chamado no inicio de eventos que
    # fazem o tempo do simulador passar (Chegada ao sistema e Saida do servidor)
    def update_n(self):
        # Calcula o intervalo de tempo entre o evento atual e o imediatamente anterior.
        delta = self.t - self.previous_event_time
        # Define o numero de pessoas nas filas de espera
        n1 = len(self.queue1)
        n2 = len(self.queue2)
        # Soma as variaveis estimadas (Nq1) e (Nq2) o numero de clientes na fila de espera multiplicado pelo
        # intervalo de tempo (delta) em que as filas ficaram com esse numero de clientes.
        self.N_samples['Nq_1'] += n1*delta
        self.N_samples['Nq_2'] += n2*delta
        # Testa se o cliente que esta no servidor, se ele estiver ocupado, veio da fila 1 ou da fila 2.
        if self.server_current_client:
            if self.server_current_client.queue == 1:
                n1 += 1
            elif self.server_current_client.queue == 2:
                n2 += 1
        # Soma as variaveis estimadas (N1) e (N2) o numero de clientes na fila multiplicado pelo
        # intervalo de tempo (delta) em que as filas ficaram com esse numero de clientes.
        self.N_samples['N_1'] += n1*delta
        self.N_samples['N_2'] += n2*delta
        # Atualiza o valor do tempo do evento anterior pelo evento atual, ja que o simulador vai processar o proximo evento.
        self.previous_event_time = self.t

    # Metodo que descarta os clientes da fase transiente e os clientes que ainda estao no sistema apos o termino do processamento
    # da rodada.
    def discard_clients(self):
        served_clients = []
        for client in self.clients:
            if client.served and client.color == EQUILIBRIUM:
                served_clients.append(client)
        self.clients = served_clients

    # Metodo que processa os dados gerados por uma rodada.
    def process_sample(self):
        s_wait_1 = 0; s_s_wait_1 = 0
        s_wait_2 = 0; s_s_wait_2 = 0
        s_server_1 = 0; s_server_2 = 0

        # Loop que faz a soma e a soma dos quadrados dos tempos de espera e a soma dos tempos em servidor
        # Dos clientes na fila 1 e na fila 2.
        for client in self.clients:
            s_wait_1 += client.wait(1)
            s_s_wait_1 += client.wait(1)**2
            s_server_1 += client.server[1]
            s_wait_2 += client.wait(2)
            s_s_wait_2 += client.wait(2)**2
            s_server_2 += client.server[2]
            
        # Adiciona a soma e a soma dos quadrados dos estimadores os valores estimados na rodada.
        self.sums['m_s_W1'] += est.mean(s_wait_1, len(self.clients))
        self.sums['m_s_s_W1'] += est.mean(s_wait_1, len(self.clients))**2
        self.sums['v_s_W1'] += est.variance(s_wait_1, s_s_wait_1, len(self.clients))
        self.sums['v_s_s_W1'] += est.variance(s_wait_1, s_s_wait_1, len(self.clients))**2
        self.sums['m_s_N1'] += est.mean(self.N_samples['N_1'], self.t)
        self.sums['m_s_s_N1'] += est.mean(self.N_samples['N_1'], self.t)**2
        self.sums['m_s_Nq1'] += est.mean(self.N_samples['Nq_1'], self.t)
        self.sums['m_s_s_Nq1'] += est.mean(self.N_samples['Nq_1'], self.t)**2
        self.sums['m_s_T1'] += est.mean(s_wait_1, len(self.clients)) + est.mean(s_server_1, len(self.clients))
        self.sums['m_s_s_T1'] += (est.mean(s_wait_1, len(self.clients)) + est.mean(s_server_1, len(self.clients)))**2
        self.sums['m_s_W2'] += est.mean(s_wait_2, len(self.clients))
        self.sums['m_s_s_W2'] += est.mean(s_wait_2, len(self.clients))**2
        self.sums['v_s_W2'] += est.variance(s_wait_2, s_s_wait_2, len(self.clients))
        self.sums['v_s_s_W2'] += est.variance(s_wait_2, s_s_wait_2, len(self.clients))**2
        self.sums['m_s_N2'] += est.mean(self.N_samples['N_2'], self.t)
        self.sums['m_s_s_N2'] += est.mean(self.N_samples['N_2'], self.t)**2
        self.sums['m_s_Nq2'] += est.mean(self.N_samples['Nq_2'], self.t)
        self.sums['m_s_s_Nq2'] += est.mean(self.N_samples['Nq_2'], self.t)**2
        self.sums['m_s_T2'] += est.mean(s_wait_2, len(self.clients)) + est.mean(s_server_2, len(self.clients))
        self.sums['m_s_s_T2'] += (est.mean(s_wait_2, len(self.clients)) + est.mean(s_server_2, len(self.clients)))**2
        # Inicializa as estruturas de dados para a proxima rodada.
        self.init_sample()
    
    # Metodo que calcula os resultados (valor e intervalo de confianca) para cada estimador.
    def calc_results(self):
        self.results = {
            'E[N1]'  : { 'value' : est.mean(self.sums['m_s_N1'], self.samples), 'c_i' : est.confidence_interval(self.sums['m_s_N1'], self.sums['m_s_s_N1'], self.samples) },
            'E[N2]'  : { 'value' : est.mean(self.sums['m_s_N2'], self.samples), 'c_i' : est.confidence_interval(self.sums['m_s_N2'], self.sums['m_s_s_N2'], self.samples) },
            'E[T1]'  : { 'value' : est.mean(self.sums['m_s_T1'], self.samples), 'c_i' : est.confidence_interval(self.sums['m_s_T1'], self.sums['m_s_s_T1'], self.samples) },
            'E[T2]'  : { 'value' : est.mean(self.sums['m_s_T2'], self.samples), 'c_i' : est.confidence_interval(self.sums['m_s_T2'], self.sums['m_s_s_T2'], self.samples) },
            'E[Nq1]' : { 'value' : est.mean(self.sums['m_s_Nq1'], self.samples), 'c_i' : est.confidence_interval(self.sums['m_s_Nq1'], self.sums['m_s_s_Nq1'], self.samples) },
            'E[Nq2]' : { 'value' : est.mean(self.sums['m_s_Nq2'], self.samples), 'c_i' : est.confidence_interval(self.sums['m_s_Nq2'], self.sums['m_s_s_Nq2'], self.samples) },
            'E[W1]'  : { 'value' : est.mean(self.sums['m_s_W1'], self.samples), 'c_i' : est.confidence_interval(self.sums['m_s_W1'], self.sums['m_s_s_W1'], self.samples) },
            'E[W2]'  : { 'value' : est.mean(self.sums['m_s_W2'], self.samples), 'c_i' : est.confidence_interval(self.sums['m_s_W2'], self.sums['m_s_s_W2'], self.samples) },
            'V(W1)'  : { 'value' : est.mean(self.sums['v_s_W1'], self.samples), 'c_i' : est.confidence_interval(self.sums['v_s_W1'], self.sums['v_s_s_W1'], self.samples) },
            'V(W2)'  : { 'value' : est.mean(self.sums['v_s_W2'], self.samples), 'c_i' : est.confidence_interval(self.sums['v_s_W2'], self.sums['v_s_s_W2'], self.samples) }
        }
    
    # Metodo que atualiza a barra de progresso com o valor do menor intervalo de confianca encontrado a cada rodada.
    def pb_amount(self):
        return 1 - max((2.0*self.results['E[N1]']['c_i']/self.results['E[N1]']['value']), \
                       (2.0*self.results['E[N2]']['c_i']/self.results['E[N2]']['value']), \
                       (2.0*self.results['E[T1]']['c_i']/self.results['E[T1]']['value']), \
                       (2.0*self.results['E[T2]']['c_i']/self.results['E[T2]']['value']), \
                       (2.0*self.results['E[Nq1]']['c_i']/self.results['E[Nq1]']['value']), \
                       (2.0*self.results['E[Nq2]']['c_i']/self.results['E[Nq2]']['value']), \
                       (2.0*self.results['E[W1]']['c_i']/self.results['E[W1]']['value']), \
                       (2.0*self.results['E[W2]']['c_i']/self.results['E[W2]']['value']), \
                       (2.0*self.results['V(W1)']['c_i']/self.results['V(W1)']['value']), \
                       (2.0*self.results['V(W2)']['c_i']/self.results['V(W2)']['value']))
            
    # Metodo que testa se todos os intervalos de confianca sao validos.
    # So faz a validacao a partir da terceira rodada.
    def valid_confidence_interval(self):
        return not(self.samples <= 2) and \
               (2.0*self.results['E[N1]']['c_i']  <= 0.1*self.results['E[N1]']['value']) and \
               (2.0*self.results['E[N2]']['c_i']  <= 0.1*self.results['E[N2]']['value']) and \
               (2.0*self.results['E[T1]']['c_i']  <= 0.1*self.results['E[T1]']['value']) and \
               (2.0*self.results['E[T2]']['c_i']  <= 0.1*self.results['E[T2]']['value']) and \
               (2.0*self.results['E[Nq1]']['c_i'] <= 0.1*self.results['E[Nq1]']['value']) and \
               (2.0*self.results['E[Nq2]']['c_i'] <= 0.1*self.results['E[Nq2]']['value']) and \
               (2.0*self.results['E[W1]']['c_i']  <= 0.1*self.results['E[W1]']['value']) and \
               (2.0*self.results['E[W2]']['c_i']  <= 0.1*self.results['E[W2]']['value']) and \
               (2.0*self.results['V(W1)']['c_i']  <= 0.1*self.results['V(W1)']['value']) and \
               (2.0*self.results['V(W2)']['c_i']  <= 0.1*self.results['V(W2)']['value'])

    # Metodo que exibe os resultados junto com o numero de rodadas processadas e os retorna.
    def report(self):
        print "Exibindo os resultados:"
        for key in self.results.keys():
            print key, ': ', self.results[key]['value'], ' - I.C: ', self.results[key]['c_i']
        print "Numero de rodadas :", self.samples
        return self.results
        
    # Metodo que inicializa a lista dos clientes que serao testados.
    def init_test(self):
        for i in range(10):
            self.test_list.append(math.floor(random.random()*self.total_clients))
    
    # Metodos que tratam o transito dos clientes das filas para o servidor, de acordo com a politica de atendimento usada.
    @staticmethod
    def pop_queue1_fcfs(instance):
        return instance.queue1.popleft()
        
    @staticmethod
    def pop_queue2_fcfs(instance):
        return instance.queue2.popleft()
        
    @staticmethod
    def pop_queue1_lcfs(instance):
        return instance.queue1.pop()
        
    @staticmethod
    def pop_queue2_lcfs(instance):
        return instance.queue2.pop()
\end{lstlisting}

\subsection{Client}
Classe que representa um cliente que entra no sistema. Possui seus tempos de entrada e saída da fila, tempo no servidor e cor.\\

\begin{lstlisting}
class Client:
    def __init__(self, color):
        # Identificador do cliente, usada para o teste de correcao.
        self.id = id    
        # Tempo de chegada ao servidor (fila 1 e fila 2)
        self.arrival = {}
        # Tempo de saida do servidor (fila 1 e fila 2)
        self.leave = {}
        # Tempo no servidor (fila 1 e fila 2)
        self.server = {}
        # Indicador que diz qual fila o cliente esta no momento
        self.queue = 0
        # Indicador que diz se o cliente ja foi servido e saiu do sistema
        self.served = 0
        # Cor do cliente (TRANSIENT e EQUILIBRIUM)
        self.color = color

    def set_arrival(self, arrival):
        self.arrival[self.queue] = arrival

    def set_leave(self, leave):
        self.leave[self.queue] = leave

    def set_server(self, server):
        self.server[self.queue] = server

    def set_queue(self, queue):
        self.queue = queue
        
    def set_served(self, served):
        self.served = served

    # Tempo de espera na fila = Tempo de saida da fila para o servidor - Tempo de chegada na fila.
    def wait(self, queue):
        return (self.leave[queue] - self.arrival[queue])
\end{lstlisting}

\subsection{EventHeap}
Classe que representa a lista de eventos que é processada durante uma rodada de simulação.\\

\begin{lstlisting}
import heapq


class EventHeap(list):
    # Adicionar evento a lista
    def push(self, (time, event_type)):
        heapq.heappush(self, (time, event_type))

    # Remover evento da lista
    def pop(self):
        return heapq.heappop(self)
\end{lstlisting}

\subsection{Analytic}
Classe que serve para calcular os resultados de forma analítica.\\

\begin{lstlisting}
from util.constants import *


class Analytic:

    def __init__(self, entry_rate, service_policy, service_rate=1.0):
        self.entry_rate = entry_rate
        self.service_policy = service_policy
        self.service_rate = service_rate
        self.utilization = 2.0*(entry_rate/service_rate)
        self.X = 1.0/self.service_rate
        self.results = { 'E[W1]'  : 0.0, 'E[W2]'  : 0.0, 'E[T1]'  : 0.0, 'E[T2]'  : 0.0,
                         'E[Nq1]' : 0.0, 'E[Nq2]' : 0.0, 'E[N1]'  : 0.0, 'E[N2]'  : 0.0,
                         'V(W1)'  : 0.0, 'V(W2)'  : 'X' }
    
    # Metodo que define os valores de forma analitica.
    def start(self):
        self.results['E[W1]']  = (self.utilization*self.X)/(1.0 - self.entry_rate*self.X)
        self.results['E[W2]']  = (self.utilization*self.results['E[W1]'] + 2.0*self.entry_rate*(self.service_rate**2))/(1.0 - self.utilization)
        self.results['E[T1]']  = self.results['E[W1]'] + self.X
        self.results['E[T2]']  = self.results['E[W2]'] + self.X
        self.results['E[Nq1]'] = self.entry_rate*self.results['E[W1]']
        self.results['E[Nq2]'] = self.entry_rate*self.results['E[W2]']
        self.results['E[N1]']  = self.entry_rate*self.results['E[T1]']
        self.results['E[N2]']  = self.entry_rate*self.results['E[T2]']
        if self.service_policy == FCFS:
            self.results['V(W1)'] = (4.0*self.utilization)/(2.0 - self.utilization)
        elif self.service_policy == LCFS:
            self.results['V(W1)'] = (4.0*self.entry_rate)*(self.entry_rate**2 - self.entry_rate + 1)/((1.0 - self.entry_rate)**3)
    
    # Metodo que exibe os resultados encontrados e os retorna.
    def report(self):
        print "Exibindo os resultados analiticos: "
        for key in self.results.keys():
            print key, ': ', self.results[key]
        
        return self.results;
\end{lstlisting}

\subsection{ResultParser}
Classe que formata os resultados encontrados em um documento .html usando o parser DOM.\\

\begin{lstlisting}
from xml.dom.minidom import *
from util.constants import *


class ResultParser:
    
    def __init__(self, results):
        self.results = results
        self.doc = Document()
    
    # Metodo que cria a estrutura html do documento e retorna o elemento <body>
    def create_header(self):
        html = self.doc.createElement('html')
        header = self.doc.createElement('header')
        title = self.doc.createElement('title')
        title_text = self.doc.createTextNode("Tabelas com os resultados da simulacao")
        body = self.doc.createElement('body')
        self.doc.appendChild(html)
        html.appendChild(header)
        html.appendChild(body)
        header.appendChild(title)
        title.appendChild(title_text)
        
        return body
    
    # Metodo que cria as tabelas
    def create_table(self, table_type, name):
        table = self.doc.createElement('table')
        table.setAttribute('cellspacing', '0')
        table.setAttribute('cellpadding', '4')
        table.setAttribute('border', '1')
        
        tr = self.doc.createElement('tr')
        th = self.doc.createElement('th')
        th.setAttribute('align', 'center')
        th.setAttribute('colspan', '31')
        th.appendChild(self.doc.createTextNode("Tabela com os resultados para a politica de atendimento " + name))
        tr.appendChild(th)
        table.appendChild(tr)
        
        tr = self.doc.createElement('tr')
        th = self.doc.createElement('th')
        th.setAttribute('align', 'center')
        th.setAttribute('style', 'font-weight: bold')
        th.appendChild(self.doc.createTextNode("uti."))
        tr.appendChild(th)
        headers = ['E[N1]', 'E[N2]', 'E[T1]', 'E[T2]', 'E[Nq1]', 'E[Nq2]', 'E[W1]', 'E[W2]', 'V(W1)', 'V(W2)']
        for header in headers:
            th = self.doc.createElement('th')
            th.setAttribute('align', 'center')
            th.appendChild(self.doc.createTextNode(header))
            tr.appendChild(th)
        table.appendChild(tr)
        
        utilizations = self.results[table_type].keys()
        utilizations.sort()
        for utilization in utilizations:
            tr = self.doc.createElement('tr')
            td = self.doc.createElement('td')
            td.setAttribute('align', 'center')
            td.setAttribute('style', 'font-weight: bold')
            td.appendChild(self.doc.createTextNode(str(utilization)))
            tr.appendChild(td)            
            for key in headers:
                td = self.doc.createElement('td')
                td.setAttribute('align', 'center')            
                div = self.doc.createElement('div')
                div.setAttribute('style', 'padding:5px;')
                if type(self.results[table_type][utilization]['analytic'][key]).__name__ == 'str':
                    div.appendChild(self.doc.createTextNode(str(self.results[table_type][utilization]['analytic'][key])))
                else:
                    div.appendChild(self.doc.createTextNode(str(round(float(self.results[table_type][utilization]['analytic'][key]), 5))))
                td.appendChild(div)
                div = self.doc.createElement('div')
                div.setAttribute('style', 'padding:5px; border-top: 1px solid #000000; border-bottom: 1px solid #000000')
                div.appendChild(self.doc.createTextNode(str(round(float(self.results[table_type][utilization]['simulator'][key]['value']), 5))))                
                td.appendChild(div)
                div = self.doc.createElement('div')
                div.setAttribute('style', 'padding:5px;')
                div.appendChild(self.doc.createTextNode(str(round(float((2.0*self.results[table_type][utilization]['simulator'][key]['c_i']/self.results[table_type][utilization]['simulator'][key]['value'])*100.0), 2)) + "%"))
                td.appendChild(div)
                tr.appendChild(td)
            table.appendChild(tr)
            
        return table
    
    # Metodo que cria o documento html usando os resultados dados
    def parse(self):
        body = self.create_header()
        table_fcfs = self.create_table(FCFS, 'F.C.F.S')
        table_lcfs = self.create_table(LCFS, 'L.C.F.S')
        table_lcfs.setAttribute('style', 'margin-top:100px')
        body.appendChild(table_fcfs)
        body.appendChild(table_lcfs)
    
    # Metodo que escreve o documento DOM gerado em um arquivo .html no disco
    def write(self, filename):
        file = open(filename, "w")
        print >>file, self.doc.toprettyxml()
        file.close()
\end{lstlisting}

\section{Modulos utilitários}

\subsection{Estimator}
Módulo que possui métodos para retornar os estimadores de média, variância e calcula intervalos de confiança.\\

\begin{lstlisting}
import math
import scipy.stats


# Retorna o valor t de student para um intervalo de confianca de 95% e [samples] amostras.
def t_st_value(samples):
    return scipy.stats.t.ppf(0.975, samples)

# Retorna a media estimada usando a soma [sum] dos valores calculados e o numero total [samples] de valores.
def mean(sum, samples):
    return sum/float(samples)

# Retorna a variancia estimada usando a forma incremental usando a soma [sum] dos valores, a soma dos quadrados [square_sum]
# e o numero total [samples] de valores.
def variance(sum, square_sum, samples):
    return square_sum/float(samples-1) - (sum**2)/float(samples*(samples-1))

# Retorna o limite do intervalo de confianca  usando a soma [sum] dos valores e a soma dos quadrados [square_sum]
# para calcular o desvio padrao e o numero de rodadas [samples].
def confidence_interval(sum, square_sum, samples):
    std_deviation = math.sqrt(variance(sum, square_sum, samples))
    return (t_st_value(samples)*std_deviation)/math.sqrt(samples)

if __name__ == "__main__":
    print "Testando..."
    list1 = [11.0, 5.0, 10.0, 9.0, 15.0, 6.0, 18.0, 8.0, 12.0, 9.0, 5.0, 10.0, 7.0, 13.0, 15.0]
    list2 = [10.0, 2.0, 15.0, 4.0, 5.0, 16.0, 8.0, 4.0, 2.0, 19.0, 10.0, 2.0, 9.0, 10.0, 12.0]
    print "Mean sample 1: ", mean(sum(list1), len(list1))
    print "Mean sample 2: ", mean(sum(list2), len(list2))
    print "Samples mean: ", mean((mean(sum(list1), len(list1)) + mean(sum(list2), len(list2))), 2)
    print "Samples variance: ", variance((mean(sum(list1), len(list1)) + mean(sum(list2), len(list2))), ((mean(sum(list1), len(list1))**2) + (mean(sum(list2), len(list2))**2)), 2)
    print "Student's T value: ", t_st_value(10000)
\end{lstlisting}

\subsection{Distribution}
Módulo com o método que retorna os tempos aleatórios de chegada de uma distribuição exponencial.\\

\begin{lstlisting}
import math
import random
import estimator


# Retorna um tempo aleatorio de uma distribuicao exponencial com taxa [rate].
def exp_time(rate):
	return -(math.log(1.0 - random.random())/rate)
	
if __name__ == "__main__":
	print "Testando..."
	print estimator.mean(exp_time(2, 650))
\end{lstlisting}

\subsection{Constants}
Módulo que declara as constantes que são utilizadas pelo simulador.\\

\begin{lstlisting}
# Constantes utilizadas para os tipos de eventos tratados pelo simulador
INCOMING = 1
SERVER_OUT = 2
SERVER_1_IN = 3
SERVER_2_IN = 4

# Constantes utilizadas para definir as cores dos clientes
TRANSIENT = 1
EQUILIBRIUM = 2

# Constantes utilizadas para definir a politica de atendimento das filas
FCFS = 1
LCFS = 2
\end{lstlisting}

\subsection{Plot}
Módulo que usa a biblioteca matplotlib para desenhar os gráficos necessários para a estimativa da fase transiente (Não é utilizado na versão final).\\

\begin{lstlisting}
from math import sin, cos
import matplotlib.pyplot as plt


def plot(list, *args, **kwargs):
    plt.plot(xrange(len(list)), list, *args, **kwargs)

def show(title):
    plt.title(title)
    plt.grid()
    plt.show()

if __name__ == "__main__":
    x = range(100)
    y = [sin(item) for item in range(100)]
    z = [cos(item) for item in range(100)]
    plot(x, y, 'b-')
    plot(x, z, 'r-')
    show()
\end{lstlisting}

\subsection{ProgressBar}
Biblioteca usada para a construção da barra de progresso usada para efeito de visualização do progresso do processamento das rodadas do simulador.

Seu código não é apresentado aqui porque ele não foi escrito por nós e os seus créditos são devidamente citados em comentários no próprio fonte.

\section{Principal}
Este é o código que executa o simulador e faz os cálculos analíticos para todo o tipo de experimento requisitado, é dado como entrada o número de clientes que serão avaliados a cada rodada. Os resultados finais gerados pelo simulador e os resultados dos cálculos analíticos são gravados em tabelas no formato .html ao final da execução.\\

\begin{lstlisting}
import time, sys, os, psyco
from obj.simulator import *
from obj.analytic import *
from obj.result_parser import *

# O psyco e um modulo que agiliza a execucao do codigo.
psyco.full()

if __name__ == "__main__":
    clients = input("Entre com o numero de clientes que serao avaliados:")
    #dados de entrada (taxa de entrada e valor da fase transiente)
    entry_data = [[0.1, 30000], [0.2, 40000], [0.3, 80000], [0.4, 400000], [0.45, 500000]]
    service_policies = [
        { 'value' : FCFS, 'name' : "F.C.F.S (First Come First Served)" },
        { 'value' : LCFS, 'name' : "L.C.F.S (Last Come First Served)"  }
    ]
    
    # Dicionario que ira guardar os resultados adquiridos pelo simulador e pelo calculo analitico
    results = {
        FCFS : {
            0.2 : { 'simulator' : {}, 'analytic' : {} },
            0.4 : { 'simulator' : {}, 'analytic' : {} },
            0.6 : { 'simulator' : {}, 'analytic' : {} },
            0.8 : { 'simulator' : {}, 'analytic' : {} },
            0.9 : { 'simulator' : {}, 'analytic' : {} }
        },
        LCFS : {
            0.2 : { 'simulator' : {}, 'analytic' : {} },
            0.4 : { 'simulator' : {}, 'analytic' : {} },
            0.6 : { 'simulator' : {}, 'analytic' : {} },
            0.8 : { 'simulator' : {}, 'analytic' : {} },
            0.9 : { 'simulator' : {}, 'analytic' : {} }
        }        
    }
    
    # Loop que ira rodar o simulador para todos os casos requeridos
    for entry_datum in entry_data:
        entry_rate = entry_datum[0]
        warm_up = entry_datum[1]
        for service_policy in service_policies:
            print "taxa de entrada =", entry_rate
            print "tamanho da fase transiente =", warm_up
            print "politica de atendimento =", service_policy['name']
            
            print "Iniciando simulacao:"
            # Chamada e execucao do simulador
            simulator = Simulator(entry_rate=entry_rate, warm_up=warm_up, clients=clients, service_policy=service_policy['value'])
            os.system("date")
            start = time.time()
            # Bind ao psyco para acelerar a execucao da logica do simulador
            psyco.bind(simulator.start)
            simulator.start()
            finish = time.time()
            print "Tempo total de execucao :", (finish - start)
            results[service_policy['value']][2.0*entry_rate]['simulator'] = simulator.report()
                                            
            print "Iniciando calculo analitico:"
            # Chamada e execucao da classe que executa os calculos analiticos
            analytic = Analytic(entry_rate=entry_rate, service_policy=service_policy['value'])
            analytic.start()
            results[service_policy['value']][2.0*entry_rate]['analytic'] = analytic.report()

    print "Gerando as tabelas com os resultados"
    # Chamada e execucao da classe que formata os resultados encontrados em um arquivo html
    parsed_result = ResultParser(results)
    parsed_result.parse()
    parsed_result.write('resultados.html')
    print "Tabelas geradas com sucesso no arquivo 'resultados.html'."
\end{lstlisting}

\section{Testes}
\label{sec:codteste}
Rotina que executa os testes de correção explicados na seção \label{sec:teste}.

\begin{lstlisting}
from obj.simulator import *


if __name__ == "__main__":
    print "Testando a correcao do simulador..."
    entry_data = [[0.1, 30000], [0.2, 40000]]
    service_policies = [
        { 'value' : FCFS, 'name' : "F.C.F.S (First Come First Served)" },
        { 'value' : LCFS, 'name' : "L.C.F.S (Last Come First Served)"  }
    ]
    clients = 100000
    
    for entry_datum in entry_data:
        entry_rate = entry_datum[0]
        warm_up = entry_datum[1]
        for service_policy in service_policies:
            print "teste com taxa", entry_rate, ", tamanho de fase transiente igual a", warm_up, ",", clients, "clientes avaliados", \
                  "e politica de atendimento", service_policy['name']
            simulator = Simulator(entry_rate=entry_rate, warm_up=warm_up, clients=clients, service_policy=service_policy['value'], test=True)
            simulator.start()
            simulator.report()
\end{lstlisting}
